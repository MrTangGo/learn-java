\section*{泛型}

\subsection*{问题的出现}

在没有出现泛型之前,如果定义一个ArryList是这样定义的。因为ArrayList中的元素是后添加,不明白添加何元素的情况下,添加元素的类型统一按照Object类添加处理。
\begin{lstlisting}[style=Java]
ArrayList arrayList = new ArrayList();
arrayList.add("aaa");
arrayList.add("bbb");
arrayList.add("ccc");
\end{lstlisting}

用迭代器来遍历这个ArryList的时候,用Object数据类型的变量来接收也是正常的。
\begin{lstlisting}[style=Java]
Iterator iterator = arrayList.iterator();
while (iterator.hasNext()){
    Object next = iterator.next();
    System.out.println(next);
}
\end{lstlisting}

但是会存在一个问题。如果使用.length()方法的话,编译会不通过。这是因为Object类没有.length()这个方法。
\begin{lstlisting}[style=Java]
next.length();
\end{lstlisting}

可是传入的元素是有字符串的,如果是一定要用.length的方法,可以使用强制转化的方法把Object类转化为String类。
\begin{lstlisting}[style=Java]
String next = (String) iterator.next();
\end{lstlisting}

这样在使用.length()的方法就不会报错了。

但是这样做也不是完美的。如果ArrayList里面添加的不仅仅只有String类型一种数据呢?如果是添加了一个Interager数据类型呢?
那还是会报错的。
\lstinputlisting[
    style=Java,
    linerange={5-19}
]{../../../generic/src/com/learnjava/generic/GenericDemo2.java}

\begin{lstlisting}[style = Java]
java.lang.ClassCastException: java.lang.Integer cannot be cast to java.lang.String
\end{lstlisting}

ArrayList可以存放任意类型的数据。例子中ArryList添加了String类型,也添加了一个Integer类型。在使用强制转化的时候(认为next都是String类型的数据,都对他进行了强身转化)程序报错了(Integer类型的数据不能强制的转化为String类型的数据)。而且这样的报错在编译的过程中还发现不了。

% 要解决上面的问题可以使用两个思路:1、可以在一开始的时候就确定好ArrayList里面的数据类型。2、也可以让ArrayList中的类型数据随着我们传入的数据类型的变化而变化。

在这个问题的背景下泛型应运而生。


\subsection*{概述}
可以使用泛型的地方有:

1、类的后面,称之为泛型类 

2、方法申明上,称之为泛型方法 

3、接口的后面,称之为泛型接口。

需要注意的规则:

规则1:
<A/B/C/D/E>:里面的字符可以按照变量的全名规则来,一般只写一个大写的英文字母,表示这是一个泛型类。

规则2:
<A,B,C...>:指定多种类型的格式,多种类型之前用逗号隔开。

\subsection*{泛型类}

类定义的过程中,类名的后面是<E>,表示这是一个泛型类。实列化泛型类的时候,必须要给这个泛型类确定具体的数据类型。

可以看看一下最常用到泛型的地方之一ArrayList的类源代码:

\begin{lstlisting}[style = Java]
public class ArrayList<E> extends AbstractList<E> implements List<E>, RandomAccess, Cloneable, java.io.Serializable{
    ...
    public E get(int index) {...}
    public E set(int index, E element) {...}
    private void add(E e, Object[] elementData, int s) {...}
}
\end{lstlisting}

类名ArrayList<E>,<E>表示接收的是泛型。在get、set方法中,返回值也是<E>类型的数据。也就是说,在实例化类的时候,返回值<E>的类型就确认好了。

\begin{lstlisting}[style = Java]
    add(E e, Object[] elementData, int s) {...}
\end{lstlisting}

同时在编译的过程中,add方法的参数<E>表明了,add方法接收的参数类型为<E>类型的数据。这样就可以在编译的过程中发现错误了。


可以尝试着自己写一个泛型类。

\lstinputlisting[
    style=Java,
    linerange={3-13}
]{../../../generic/src/com/learnjava/generic/GenericDemo4.java}

\lstinputlisting[
    style=Java,
    linerange={3-13}
]{../../../generic/src/com/learnjava/generic/Box.java}






\subsection*{泛型方法}
