\section*{异常}


\subsection*{throws方式处理异常}
在方法声明的末端throws出去,把异常抛给方法的调用者。
\begin{lstlisting}[style=Java]
public void 方法() throws 异常类名 {

}
\end{lstlisting}

\lstinputlisting[
    style=Java
]{../../../exception/src/com/learnjava/exception/ThrowsDemo1.java}

注意:这个throws格式是跟在方法的括号后面的。编译时异常必须要进行处理,两种处理方案:try...catch …或者 throws,如果采用 throws 这种方案,在方法上进行显示声明,将来谁调用这个方法谁处理。运行时异常因为在运行时才会发生,所以在方法后面可以不写,运行时出现异常默认交给jvm处理


